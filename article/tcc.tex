\documentclass[12pt]{article}

\usepackage{sbc-template}
\usepackage{graphicx,url}
\usepackage[utf8]{inputenc}
\usepackage[brazil]{babel}
\usepackage{fancyvrb}
\usepackage{caption}
\usepackage{subcaption}

\newcommand\slsh{\char`\\}

\usepackage{hyperref}
\hypersetup{
    colorlinks=true,
    linkcolor=cyan,
    filecolor=blue,      
    urlcolor=cyan,
    citecolor=cyan,
}
 
\sloppy

\title{Estudo relativo ao desempenho em função do consumo sobre alta demanda de dados em sistemas web desenvolvidos com \textit{Node.JS} e \textit{React.JS}}

\author{Marcos Renan Krul\inst{1}, Renato Cristiano Ruppel\inst{1}, Prof. Dr. Adriano Ferrsa\inst{1}}


\address{Universidade Estadual de Ponta Grossa (UEPG)
    \email{19022626@uepg.br, 19010426@uepg.br, ferrasa@uepg.br}
}


\begin{document} 

\maketitle


\begin{resumo} 
\end{resumo}


\begin{abstract} 
\end{abstract}


\section{Introdução}

O desenvolvimento de novas aplicações e a migração das já existentes para a plataforma \textit{web} 
vêm crescendo e se tornando uma tendência \cite{SOUZAB}. Existem benefícios que justificam esse comportamento,
como facilidades em instalações e atualizações, alcance global instantâneo e alta portabilidade. Porém,
sob outra perspectiva, apresentam-se questões críticas que devem ser analisadas antes de aderir à plataforma, como
o desempenho em situações com alta demanda de dados.

Aplicações como \textit{streaming} de vídeo (ADICIONAR OUTROS EXEMPLOS) devem processar arquivos de 
grande porte, de forma rápida e estável. Contudo, existem limitações inerentes ao uso da 
plataforma \textit{web}, como o processamento de diversas requisições que exijam muitos dados, 
tratamento do alto fluxo de dados pela máquina cliente e os limites impostos pelas tecnologias escolhidas.

Para sobrepujar os problemas supracitados, existem técnicas desenvolvidas para a \textit{web} que buscam
atenuar os problemas de alta demanda e torná-los irrisórios na experiência do usuário, como o processamento
assíncrono (\textit{streams}) e \textit{caching}.

O uso de \textit{Node.JS} e \textit{React.JS}, duas tecnologias em crescente uso e relevância no
desenvolvimento \textit{web}, justifica-se pelo fato de que estas permitem uma abordagem de alta
performance e apresentam recursos para otimização e escalabilidade. O \textit{Node.JS}, em específico, apresenta
bons resultados em casos que necessitem da persistência de centenas ou milhares de conexões
simultâneas, onde a comunicação é realizada com o envio de pequenos fragmentos do arquivo de destino
\cite[p. 112]{EJSMONT}. Quanto à camada \textit{front-end}, a biblioteca \textit{React.JS} traz como
carro chefe o gerenciamento de estado da aplicação, orientado a dados que podem mudar com o passar do tempo.
Ao incluir estado em uma aplicação utilizando \textit{React.JS}, inclui-se a possibilidade de criar, ler
e modificar dados dinâmicos enviados pelo servidor, que modificam a árvore de componentes e acarretam 
mudanças na interface em que o usuário interage \cite[p. 97]{BANKSEPORCELLO}.


% Caching is one of the most important techniques when it comes to scaling the
% front end of your web application. Instead of trying to add more servers or make
% them respond faster to clients’ requests, use caching to avoid having to serve
% these requests in the first place. In fact, caching is so critical to the scalability of
% web applications that Chapter 6 is dedicated to it. To avoid repeating parts of that
% deeper dive, let’s just highlight a few components relevant to the front-end layer
% of your application here. (p 133 livro);


% À medida que a computação como um todo evoluiu, desde a comunicação
% utilizando apenas bits, para a comunicação com dados mais complexos e robustos,
% a criação e implementação de novas técnicas que adaptem a manipulação de novas
% formas de dados mantendo performance, simplicidade e robustez, torna-se um
% desafio.

% Todo projeto de software precisa de estimativas e planejamento, mas quando
% tem-se um sistema desenvolvido para a web, principalmente onde exista grandes
% volumes de dados, o termo escalabilidade deve se tornar muito mais presente na
% concepção do projeto.

% O termo escalabilidade pode ser definido como a maneira de ajustar a
% capacidade de um sistema para atender demandas, de forma que o custo desta
% operação não seja tão elevado. \cite[p. 3]{EJSMONT}. A incorporação de
% escalabilidade em um sistema pode compreender a capacidade de lidar com um
% crescimento de usuários, solicitações e transações, para este projeto, seu uso será
% voltado para a requisição e transferência de grandes volumes de dados.


\section{Revisão de bibliografia e trabalhos relacionados}


\section{Metodologia}


\section{Análises preliminares}


\section{Considerações finais}


\section{Referências}
\bibliographystyle{sbc}
\bibliography{referencias}

\end{document}

% Resumo
% Introdução (Contexto do tema, problemática, justificativa e objetivos do estudo)
% Revisão da bibliografia e trabalhos relacionados
% Metodologia (Materiais e/ ou métodos utilizados)
% Análises preliminares
% Considerações finais (sobre o que já foi apresentado até o momento da entrega desta versão)